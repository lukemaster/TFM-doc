\begin{longtable}{|>{\columncolor[rgb]{0.75,0.75,0.75}}p{3cm}|p{11cm}|}
\caption{Caso de uso \emph{GenerarPiezaMusical.}} \\
\hline \centerline{\textcolor[rgb]{1.00,1.00,1.00}{\textbf{\small Nombre}}} & {\small GenerarPiezaMusical.}
\\
\hline \centerline{\textcolor[rgb]{1.00,1.00,1.00}{\textbf{\small Descripción}}} & {\small Representación de todas las acciones necesarias para generar una pieza musical.}
\\
\hline \centerline{\textcolor[rgb]{1.00,1.00,1.00}{\textbf{\small Actores}}} & {\small Usuario.}
\\
\hline
\begin{center}
\textcolor[rgb]{1.00,1.00,1.00}{\textbf{\small Casos de uso}}
\end{center}
\begin{center}

\end{center}
& {\small \emph{SolicitarPiezaMusical:} permite al usuario solicitar la generación de una
pieza musical.}

{\small \emph{SeleccionarGénero:} brinda al usuario la posibilidad de seleccionar el género de la pieza musical a generar.}

\\
\hline
\begin{center}
\end{center}
\begin{center}
\textcolor[rgb]{1.00,1.00,1.00}{\textbf{\small Flujo de eventos
principal}}
\end{center}
& {\small Pasos de ejecución del camino básico del caso de uso:}

{\small
\begin{enumerate}
    \item El usuario accede al sistema.

    \begin{enumerate}
        \item[] 1.1 El usuario selecciona un género musical de una lista dada.
        \item[] 1.2 El usuario solicita una muestra de música generada, asociada al género musical seleccionado.
        \item[] 1.3 El sistema devuelve la pieza musical generada.
    \end{enumerate}
\end{enumerate}
}
\\
\hline \centerline{\textcolor[rgb]{1.00,1.00,1.00}{\textbf{\small Flujo de eventos}}}
\centerline{\textcolor[rgb]{1.00,1.00,1.00}{\textbf{\small excepcional}}} & {\small No se contempla.}
\\
\hline
\end{longtable}