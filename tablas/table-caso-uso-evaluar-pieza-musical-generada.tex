\begin{longtable}{|>{\columncolor[rgb]{0.75,0.75,0.75}}p{3cm}|p{11cm}|}
\caption{Caso de uso \emph{EvaluarPiezaMusicalGenerada.}} \\
\hline \centerline{\textcolor[rgb]{1.00,1.00,1.00}{\textbf{\small
Nombre}}} & {\small EvaluarPiezaMusicalGenerada.}
\\
\hline \centerline{\textcolor[rgb]{1.00,1.00,1.00}{\textbf{\small
Descripción}}} & {\small Representación de las acciones necesarias para evaluar la pieza musical generada.}
\\
\hline \centerline{\textcolor[rgb]{1.00,1.00,1.00}{\textbf{\small
Actores}}} & {\small Usuario.}
\\
\hline
\begin{center}
\textcolor[rgb]{1.00,1.00,1.00}{\textbf{\small Casos de uso}}
\end{center}
\begin{center}

\end{center}
& {\small \emph{PuntuarPiezaMusical:} permite al usuario puntuar la pieza musical generada en base a su percepción subjetiva de pertenencia al género solicitado.}

{\small \emph{GenerarPiezaMusical:} permite al usuario solicitar la genearción de una pieza musical, dado un género concreto y descargarla.}

\\
\hline
\begin{center}
\end{center}
\begin{center}
\textcolor[rgb]{1.00,1.00,1.00}{\textbf{\small Flujo de eventos
principal}}
\end{center}
& {\small Pasos de ejecución del camino básico del caso de uso:}

{\small
\begin{enumerate}
    \item  El usuario solicita la creación de una pieza musical de un género concreto.

    \item  Una vez obtenida, el usuario puede emitir una valoración sobre la pertenencia de la pieza al género solicitado.
\end{enumerate}
}
\\
\hline \centerline{\textcolor[rgb]{1.00,1.00,1.00}{\textbf{\small
Flujo de eventos}}}
\centerline{\textcolor[rgb]{1.00,1.00,1.00}{\textbf{\small
excepcional}}} & {\small El paso número 2 es opcional.}
\\
\hline
\end{longtable}