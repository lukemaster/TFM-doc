% \begin{table}[h]
%     \centering
    
%     \resizebox{\textwidth}{!}{
%     \begin{tabular}{|c|c|p{8cm}|c|}
%         \hline
%         \textbf{Épica} & \textbf{Historia de Usuario} & \textbf{Tareas} & \textbf{Prioridad} \\
%         \hline
%         Comprensión del negocio & Definir el problema & Redactar introducción, revisar 10 artículos, definir impacto y KPIs & Alta \\
%         \hline
%         Comprensión de los datos & Analizar calidad de los datos & Inspeccionar valores nulos, outliers, generar gráficos EDA & Alta \\
%         \hline
%         Preparación de los datos & Preprocesamiento de datos & Normalizar, codificar variables, dividir en train/test & Media \\
%         \hline
%         Modelado & Entrenar modelos de ML & Implementar baseline, probar algoritmos, ajustar hiperparámetros & Alta \\
%         \hline
%         Evaluación & Validar rendimiento del modelo & Comparar métricas, generar reportes de resultados & Alta \\
%         \hline
%         Implementación y monitoreo & Redactar el TFM & Documentar metodología, resultados, preparar presentación & Alta \\
%         \hline
%     \end{tabular}}
% \end{table}

\begin{table}[h]
    \centering
    \caption{Product Backlog (Pila de producto) del TFM basado en CRISP-ML(Q) y Scrum}
    \resizebox{0.8\textwidth}{!}{
    \begin{tabular}{|c|p{8cm}|p{8cm}|}
        \hline
        \textbf{Épica (Fase CRISP-ML(Q))} & \textbf{Historia de Usuario} & \textbf{Tareas} \\
        \hline
        Comprensión del negocio &  Como estudiante, quiero comprender bien el problema para entender el alcance del proyecto. & 
        \begin{itemize}
            \item Redactar introducción.
            \item Definir objetivos generales y específicos.
            \item Entender y tomar notas sobre el alcance del proyecto.
        \end{itemize} \\
        \hline
        Comprensión del negocio & Como estudiante, en mi labor de investigación, quiero revisar literatura existente en torno al problema.  &
        \begin{itemize}
            \item Buscar y analizar artículos y publicaciones relevantes. 
            \item Extraer ideas clave.
            \item Identificar tendencias en el estado del arte.
        \end{itemize} \\
        \hline
        Comprensión de los datos & Como estudiante, emulando a un científico de datos, quiero inspeccionar los datos disponibles para evaluar su calidad. & 
        \begin{itemize}
            \item Encontrar fuentes con datasets adecuados.
            \item Revisar estructura y tipos de datos.
            \item Evaluar insuficiencia, sesgo o desbalanceo en los datos.
            \item Escribir la metodología detallada.
        \end{itemize} \\
        \hline
        Preparación de los datos & Como estudiante y desarrollador, quiero limpiar y transformar los datos para que sean aptos para el modelo. & 
        \begin{itemize}
            \item Normalizar y codificar variables.
            \item Manejar valores atípicos y nulos.
        \end{itemize} \\
        \hline
        Preparación de los datos & Como estudiante y desarrollador, quiero dividir los datos en conjuntos de entrenamiento, validación y prueba. & 
        \begin{itemize}
            \item Definir proporción de separación de datos.
            \item Verificar balanceo de clases.
        \end{itemize} \\
        \hline
        Modelado & Como estudiante, emulando a un científico de datos, quiero entrenar un modelo base para establecer un punto de comparación. & 
        \begin{itemize}
            \item Entrenar el VAE.
            \item Evaluar rendimiento inicial.
            \item Extraer los primeros resultados.
        \end{itemize} \\
        \hline
        Modelado & Como estudiante, emulando a un científico de datos, quiero entrenar un modelo basado en GAN-Transformer y evaluar su precisión. & 
        \begin{itemize}
            \item Implementar y entrenar el modelo GAN-Transformer.
            \item Comparar modelos con métricas clave.
        \end{itemize} \\
        \hline
        Evaluación & Como estudiante e investigador, quiero evaluar el rendimiento del modelo para justificar su efectividad. &
        \begin{itemize}
            \item Calcular precisión, recall y F1-score.
            \item Generar matriz de confusión.
            \item Comparar modelos.
        \end{itemize} \\
        \hline
        Evaluación & Como estudiante e investigador, quiero establecer métricas comparativas entre los enfoques y mejorar la red GAN. & 
        \begin{itemize}
            \item Analizar interpretabilidad de los resultados.
            \item Ajustar hiperparámetros y arquitectura.
        \end{itemize} \\
        \hline
        Implementación y monitoreo & Como estudiante, quiero documentar la metodología utilizada para su presentación y evaluación. & 
        \begin{itemize}
            \item Documentar pruebas de los modelos.
            \item Preparar informe técnico.
        \end{itemize} \\
        \hline
        Implementación y monitoreo & Como estudiante, quiero preparar una presentación clara y concisa para la defensa del TFM. & 
        \begin{itemize}
            \item Diseñar diapositivas con gráficos y métricas.
            \item Exponer conclusiones.
            \item Practicar la presentación.
        \end{itemize} \\
        \hline
    \end{tabular}}
\end{table}
