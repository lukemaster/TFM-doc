\begin{tcolorbox}[colback=black!85!white, colframe=orange!60!black, fontupper=\color{white},
title=\textbf{\huge Kanban}, fonttitle=\color{white}]

{\huge \emph{Sprint 3}}
    \begin{center}
    \resizebox{0.8\textwidth}{!}{\begin{tabular}{m{4cm} m{4cm} m{4cm}}
        \begin{center}
        {\Fontauri \huge \emph{to do}}
        \end{center}
        & \begin{center}
        {\Fontauri \huge \emph{in progress}} 
        \end{center} & 
        \begin{center}{\Fontauri \huge \emph{done}} 
        \end{center} \\
        
        \begin{tcolorbox}[colback=green!50!white, colframe=green!80!black,
            title={\textbf{PD-2}}, 
            fonttitle=\color{black}
        ]
        Normalización y codificación de variables.
        \end{tcolorbox} & \begin{tcolorbox}[colback=purple!30!white, colframe=purple!40!white,
            title={\textbf{CD-3}}, 
            fonttitle=\color{black}
        ]
        Identificar valores nulos y outliers.
        \end{tcolorbox} 
        &
        \resizebox{0.8\textwidth}{!}{\begin{tabular}{m{4cm} m{4cm} m{4cm}}
        \begin{tcolorbox}[colback=yellow!70!white, colframe=yellow!90!white,
            title={\textbf{CN-1}}, 
            fonttitle=\color{black}
        ]
        \vspace{0.3cm}
        Definir problema del TFM.
        \end{tcolorbox}
        & 
        \begin{tcolorbox}[colback=yellow!70!white, colframe=yellow!90!white,
            title={\textbf{CN-2}}, 
            fonttitle=\color{black}
        ]
        Revisar al menos 10 artículos.
        \end{tcolorbox} 
        & 
        \begin{tcolorbox}[colback=yellow!70!white, colframe=yellow!90!white,
            title={\textbf{CN-3}}, 
            fonttitle=\color{black}
        ]
        Definir métricas clave (KPIs).
        \end{tcolorbox} 
        \\
        \end{tabular}
        }
        

        \\
        \begin{tcolorbox}[colback=green!50!white, colframe=green!80!black,
            title={\textbf{PD-3}}, 
            fonttitle=\color{black}
        ]
        Dividir dataset en train/val/test.
        \end{tcolorbox} & \begin{tcolorbox}[colback=green!50!white, colframe=green!80!black,
            title={\textbf{PD-1}}, 
            fonttitle=\color{black}
        ]
        Limpieza y preprocesamiento de datos.
        \end{tcolorbox} &
        
        \\
        \begin{tcolorbox}[colback=pink!60!red, colframe=pink!40!red,
            title={\textbf{M-1}}, 
            fonttitle=\color{black}
        ]
        Implementar modelo base (benchmark).
        \end{tcolorbox} & &
        
        \\
        
        \begin{tcolorbox}[colback=pink!60!red, colframe=pink!40!red,
            title={\textbf{M-2}}, 
            fonttitle=\color{black}
        ]
        Probar al menos 2-3 modelos.
        \end{tcolorbox} & &
        \begin{tcolorbox}[colback=purple!30!white, colframe=purple!40!white,
            title={\textbf{CD-1}}, 
            fonttitle=\color{black}
        ]
        \vspace{0.3cm}
        Recopilar datasets disponibles.
        \end{tcolorbox} 
        \\
        \begin{tcolorbox}[colback=pink!60!red, colframe=pink!40!red,
            title={\textbf{M-3}}, 
            fonttitle=\color{black}
        ]
        Ajustar hiperparámetros con GridSearch/Optuna.
        \end{tcolorbox} & &
        \begin{tcolorbox}[colback=purple!30!white, colframe=purple!40!white,
            title={\textbf{CD-2}}, 
            fonttitle=\color{black}
        ]
        Realizar análisis exploratorio (EDA).
        \end{tcolorbox} 
        \\
        \begin{tcolorbox}[colback=orange!70!white, colframe=orange!90!white,
            title={\textbf{Ev-1}}, 
            fonttitle=\color{black}
        ]
        Comparar métricas de rendimiento.
        \end{tcolorbox} & &
        \begin{tcolorbox}[colback=purple!30!white, colframe=purple!40!white,
            title={\textbf{CD-4}}, 
            fonttitle=\color{black}
        ]
        Generar visualización de datos.
        \end{tcolorbox}
        \\        
        \begin{tcolorbox}[colback=orange!70!white, colframe=orange!90!white,
            title={\textbf{Ev-2}}, 
            fonttitle=\color{black}
        ]
        Generar matriz de confusión.
        \end{tcolorbox}  &
        
        &
        \\
        \begin{tcolorbox}[colback=blue!30!white, colframe=blue!40!white,
            title={\textbf{IM-1}}, 
            fonttitle=\color{black}
        ]
        Preparar presentación final.
        \end{tcolorbox} & & \\
    
    \end{tabular}}
    \end{center}

\end{tcolorbox}