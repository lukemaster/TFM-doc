
\subsection{Diagramas de Secuencia}

Tras la especificación del modelo de clases, el siguiente paso es crear un nuevo diagrama, en el que los dos hasta ahora mostrados, el diagrama de caso de uso y la especificación de modelo de clases, se unan y muestren una nueva visión de la información, en la que se aprecie el flujo de información entre las clases, recreando las funciones de los casos de uso, y mostrando todo esto bajo una progresión temporal. A esto se le llama Diagrama de Secuencia o Diagrama de Caso de Uso Dinámico.

En estos diagramas, la progresión temporal se despliega en el eje vertical hacia abajo y el intercambio de datos entre clases se representa con líneas horizontales que van desde un usuario a una clase o de una clase a otra.

Los diagramas de secuencia diseñados son los siguientes:

\begin{itemize}
    \item Creación de pieza musical.
    \item Evaluación de pieza musical.
    \item Ayuda.
\end{itemize}

Estos diagramas de secuencia dejan entrever que, en todo momento, la instanciación de objetos de clase ya se ha producido. Es decir, el sistema será un software latente e instanciable bajo demanda, que carecerá de necesidad de ser inicializado de cualquier modo.

\newpage
\subsubsection{Creación de pieza musical}

Este diagrama de secuencia representa el flujo de eventos que ocurren cuando se solicita la creación de una pieza musical.

El usuario insta al sistema, ya iniciado previamente y este le da la oportunidad de selecionar el género musical.

Tras esta selección, se produce la llamada al objeto ModeloGeneroMúsica, que devolverá una pieza musical del género requerido.

\begin{figure}[H]
  \centering
  \includesvg[width=0.9\textwidth]{images/diagrama-secuencia-generar.svg}
  \caption{Diagrama de secuencia \emph{Creación de pieza musical}.}
\end{figure}

\subsubsection{Evaluación de pieza musical}

Este diagrama de secuencia representa el flujo de eventos que ocurren cuando se emite una evaluación sobre una pieza musical generada.

El usuario insta al sistema y recorre todos los pasos a seguir para la generación de una pieza musical de un género concreto. Una vez generada, ya puede emitir su valoración. El sistema albergará dicha evaluación.

\begin{figure}[H]
  \centering
  \includesvg[width=0.9\textwidth]{images/diagrama-secuencia-eval.svg}
  \caption{Diagrama de secuencia \emph{Evaluación de pieza musical}.}
\end{figure}

\subsubsection{Consultar ayuda}

Este diagrama de secuencia representa el flujo de eventos que ocurren cuando se consulta la ayuda sobre el uso del programa.

El usuario, en cualquier momento, puede solicitar ayuda sobre el uso y funcionamiento del sistema al mismo. El sistema muestra la ayuda del programa por pantalla.

\begin{figure}[H]
  \centering
  \includesvg[width=0.5\textwidth]{images/diagrama-secuencia-ayuda.svg}
  \caption{Diagrama de secuencia \emph{Consultar ayuda}.}
\end{figure}