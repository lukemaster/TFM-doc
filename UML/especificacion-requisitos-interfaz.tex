
\subsection{Especificación de Requisitos de la Interfaz}
\label{especificacion-interfaz}
Hoy en día no se concibe un programa informático sin una interfaz gráfica de usuario.

La interfaz gráfica de usuario (GUI) brinda al usuario la capacidad de introducir y visualizar información de manera sencilla e intuitiva, y permite interaccionar con una aplicación de manera más amigable.

Además de esto, las GUI son una parte fundamental en una sociedad en que la imagen externa se antepone a la potencialidad o valor oculto tras esa máscara. De ahí la expresión: ``se mete por los ojos''. Algo que ``entra por los ojos'' es algo que cala de tal manera que tiene más probabilidad de ser aceptado que lo que no nos parece atractivo.

La GUI también permite el acceso a la informática a personas cuyas discapacidades dificultan o impiden el manejo de un antiguo programa de consola, o también da acceso a un colectivo social como es el colectivo infantil, al que permanecer delante de algo que no estimula su atención le resulta muy difícil.

Puesto que esta aplicación va orientada al uso por parte del mayor número de usuarios, con perfiles personales y profesionales distintos, la interfaz gráfica ha de ser lo más sencilla y amigable posible.

El programa se compondrá de una única ventana principal, cuyo contenido será:

\begin{itemize}
    \item Una lista de géneros musicales para seleccionar.
    \item Un botón para instar la creación de una pieza musical del género seleccionado.
    \item Un botón para dirigirse a la zona de ayuda de la ventana.
    \item Un pequeño panel que se activará cuando se recepciones el fichero con la pieza musical y que permitirá al usuario emitir una valoración entre 1 y 10 de la fidelidad de dicha pieza con el género solicitado.
\end{itemize}