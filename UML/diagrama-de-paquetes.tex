
\subsection{Diagrama de Paquetes}

La complejidad y tamaño de un diagrama de clases de un sistema puede llevar que la aprehensión del mismo sea compleja y difusa.

En la programación estructurada, la manera de crear un nivel más alto de abstracción es la creación de las funciones. Pero en la programación orientada a objetos, donde los datos y el procesamiento no se presentan por separado, la única manera de realizar un diagrama más inteligible y abordable, así como más sencillo de entender es el diagrama de paquetes.

Un paquete es una estructura superior, mediante la cual se agrupan varias clases, atendiendo a un criterio de empaquetamiento. El criterio usado para crear el diagrama de paquetes de este proyecto es el desempeño. Así, todas las clases utilizadas para un misma finalidad estarán agrupadas bajo un mismo paquete.

Los paquetes que se pueden extraer de la sección ``Diseño de clases''\ref{diseño-clases} son los siguientes:

\begin{itemize}
    \item Aplicación.
    \item CargadorDataset.
    \item ModelosIA.
\end{itemize}

No existirá jerarquía entre paquetes, por lo tanto, ningún paquete contendrá a uno o varios otros.

\subsubsection{Paquete Aplicación}
Este paquete contiene las clases encargadas de conformar una aplicación informática de cara al usuario. 

Las clases contenidas son:

\begin{itemize}
  \item Sistema.
  \item SelectorGéneroMusical.
  \item EvaluadorPiezaGenerada.
  \item Ayuda.
\end{itemize}

La clase \emph{Sistema} y por ende, este paquete, verán al modelo de Inteligencia Artificial ya compilado como recurso.

\subsubsection{Paquete ModelosIA}

Este otro paquete es el encargado de contener todas las clases que intervienen en la conformación de un modelo de Inteligencia Artificial y que dará lugar a los objetos de sendos modelos utilizados en este trabajo.

La lista de clases que se pueden encontrar en este paquete son:
\begin{itemize}
  \item modeloGeneradorIA.
  \item VAE.
  \item GAN.
  \item Transformer.
\end{itemize}


\subsubsection{Paquete CargadorDataset}

Las clases que están contenidas en este paquete van orientadas al manejo del \emph{Dataset} y la abstracción de las piezas musicales en él contenidas.

Las clases que conforman este paquete son:
\begin{itemize}
  \item PiezaMusical.
  \item PiezaMusicalEtiquetada.
  \item CargadorMP3.
  \item ScriptEntrenamiento.
\end{itemize}

Se hace un especial hincapié en la clase \emph{ScriptEntrenamiento}, que si bien no está colocada en este paquete de manera trivial, pues sirve para manejar el \emph{Dataset}, también controla y gestiona todo el proceso de entrenamiento de los modelos y es la pieza central de todo el proceso previa puesta en producción, que se hace para generarlos.


\subsubsection{Diagrama general de paquetes}

\begin{figure}[H]
  \centering
  \includesvg[width=0.4\textwidth]{images/diagrama-paquetes.svg}
  \caption{Diagrama general de paquetes.}
\end{figure}
