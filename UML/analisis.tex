\subsection{Análisis del sistema}
\label{analisis-del-sistema}

Se ha elegido UML como método de diseño del software fruto de este Trabajo de Fin de Máster.

UML es un lenguaje de modelado de software que tiene sus propios mecanismos para llevar a cabo las tareas, desde la definición de lo
que serán los ``requisitos funcionales'': lo que se pretende que sea capaz de proporcionar el programa al usuario; hasta la definición de las clases que intervendrán en la codificación de la aplicación, así como cada flujo de información contemplado en el programa final.

\subsubsection{Formalización de la especificación de requisitos}

Los requisitos funcionales representan un manifiesto de todos los servicios que el programa proporcionará al usuario y cómo se comporta el sistema en cada una de las acciones llevadas a cabo para rendir esos servicios.

Los requisitos funcionales están denotados por la letra F, seguida de un guión y un número de requisito que lo clasificará como único en el sistema. Los requisitos funcionales de la aplicación son los siguientes:

\begin{itemize}

    \item F-1: el sistema debería dar la oportunidad al usuario de descargar una pieza musical según un género concreto:
    \begin{itemize}
        \item F-1.1: el sistema debería mostrar una lista de géneros musicales y dar la oportunidad al usuario de seleccionar uno de ellos.
        \item F-1.2: el sistema debería dar la oportunidad al usuario de instar la generación de una pieza musical que concuerde con el género seleccionado.
        \item F-1.3: el sistema debería dar la oportunidad de descargar la pieza musical generada
    \end{itemize}
    
    \item F-2: el sistema debería proporcionar la oportunidad de evaluar la pieza musical generada.
    \begin{itemize}
        \item F-2.1: el sistema debería proporcionar una manera de almacenar datos de cada evaluación.
    \end{itemize}
       
    \item F-3: el sistema debería dar la oportunidad al usuario de consultar una ayuda sobre el uso y comportamiento del programa.

\end{itemize}

\subsubsection{Diagramas de Casos de Uso del Sistema}

El siguiente paso a seguir en el proceso definido por UML es el diseño de casos de uso.

Los casos de uso representan todas las acciones que pueden realizar uno o más usuarios en cada momento con la aplicación que se modela. A continuación se definen cuáles son los diagramas de casos de uso del sistema.

Los casos de uso que se van describir son los siguientes:


\begin{itemize}

    \item PresentaciónDelSistema. Diagrama 1.

    \item GenerarPiezaMusical. Diagrama 2

    \item EvaluarPiezaMusicalGenerada. Diagrama 3.

\end{itemize}

\begin{enumerate}

\item{\textbf{Diagrama de Casos de Uso: PresentaciónDelSistema}}

\begin{figure}[H]
  \centering
  \includesvg[width=0.55\textwidth]{images/caso-uso-presentaciondelsistema.svg}
  \caption{Diagrama de caso de uso principal del sistema.}
  \label{fig:caso-uso-presentaciondelsistema}
\end{figure}

\begin{longtable}{|>{\columncolor[rgb]{0.75,0.75,0.75}}p{3cm}|p{11cm}|}
\hline \centerline{\textcolor[rgb]{1.00,1.00,1.00}{\textbf{\small Nombre}}} & {\small Presentación del sistema.}
\\
\hline \centerline{\textcolor[rgb]{1.00,1.00,1.00}{\textbf{\small
Descripción}}} & {\small Establecimiento de las relaciones entre los elementos internos del sistema y la interacción con los elementos externos, describiéndose la presentación principal del mismo.}
\\
\hline
\centerline{\textcolor[rgb]{1.00,1.00,1.00}{\textbf{\small
Actores}}}
&
{\small Usuario.}
\\
\hline
\begin{center}
\textcolor[rgb]{1.00,1.00,1.00}{\textbf{\small Casos de uso}}
\end{center}
\begin{center}

\end{center}
&

{\small \emph{GenerarPiezaMusical:} Permite al usuario solicitar la generación de una pieza musical.}

{\small \emph{ConsultarAyuda:} Permite al usuario consultar la ayuda
sobre el funcionamiento del programa.}
\\
\hline
\begin{center}
\end{center}
\begin{center}
\textcolor[rgb]{1.00,1.00,1.00}{\textbf{\small Flujo de eventos
principal}}
\end{center}
&
{\small Pasos de ejecución del camino básico del caso de uso:}

{\small
\begin{enumerate}
    \item El usuario accede al sistema.

    \item El sistema muestra las opciones que se pueden llevar a cabo.

    \item El usuario elige una de las opciones.
\end{enumerate}
}
\\
\hline \centerline{\textcolor[rgb]{1.00,1.00,1.00}{\textbf{\small
Flujo de eventos}}}
\centerline{\textcolor[rgb]{1.00,1.00,1.00}{\textbf{\small
excepcional}}} & {\small No se contempla.}
\\
\hline
\end{longtable}

\item{\textbf{Diagrama de Casos de Uso: GenerarPiezaMusical}}

\begin{figure}[H]
  \centering
  \includesvg[width=0.6\textwidth]{images/caso-uso-generarpiezamusical.svg}
  \caption{Diagrama de caso de uso de generación de una pieza musical.}
  \label{fig:caso-uso-generarpiezamusical}
\end{figure}

\begin{longtable}{|>{\columncolor[rgb]{0.75,0.75,0.75}}p{3cm}|p{11cm}|}
\hline \centerline{\textcolor[rgb]{1.00,1.00,1.00}{\textbf{\small Nombre}}} & {\small GenerarPiezaMusical.}
\\
\hline \centerline{\textcolor[rgb]{1.00,1.00,1.00}{\textbf{\small Descripción}}} & {\small Representación de todas las acciones necesarias para generar una pieza musical.}
\\
\hline \centerline{\textcolor[rgb]{1.00,1.00,1.00}{\textbf{\small Actores}}} & {\small Usuario.}
\\
\hline
\begin{center}
\textcolor[rgb]{1.00,1.00,1.00}{\textbf{\small Casos de uso}}
\end{center}
\begin{center}

\end{center}
& {\small \emph{SolicitarPiezaMusical:} permite al usuario solicitar la generación de una
pieza musical.}

{\small \emph{SeleccionarGénero:} brinda al usuario la posibilidad de seleccionar el género de la pieza musical a generar.}

\\
\hline
\begin{center}
\end{center}
\begin{center}
\textcolor[rgb]{1.00,1.00,1.00}{\textbf{\small Flujo de eventos
principal}}
\end{center}
& {\small Pasos de ejecuci�n del camino b�sico del caso de uso:}

{\small
\begin{enumerate}
    \item El usuario accede al sistema.

    \begin{enumerate}
        \item[] 1.1 El usuario selecciona un género musical de una lista dada.
        \item[] 1.2 El usuario solicita una muestra de música generada, asociada al género musical seleccionado.
        \item[] 1.3 El sistema devuelve la pieza musical generada.
    \end{enumerate}
\end{enumerate}
}
\\
\hline \centerline{\textcolor[rgb]{1.00,1.00,1.00}{\textbf{\small Flujo de eventos}}}
\centerline{\textcolor[rgb]{1.00,1.00,1.00}{\textbf{\small excepcional}}} & {\small No se contempla.}
\\
\hline
\end{longtable}

\item{\textbf{Diagrama de Casos de Uso: EvaluarPiezaMusicalGenerada}}

\begin{figure}[H]
  \centering
  \includesvg[width=0.58\textwidth]{images/caso-uso-evaluarpiezamusicalgenerada.svg}
  \caption{Diagrama de caso de uso de evaluación de una pieza musical.}
  \label{fig:caso-uso-evaluarpiezamusicalgenerada}
\end{figure}

\begin{longtable}{|>{\columncolor[rgb]{0.75,0.75,0.75}}p{3cm}|p{11cm}|}
\hline \centerline{\textcolor[rgb]{1.00,1.00,1.00}{\textbf{\small
Nombre}}} & {\small EvaluarPiezaMusicalGenerada.}
\\
\hline \centerline{\textcolor[rgb]{1.00,1.00,1.00}{\textbf{\small
Descripci�n}}} & {\small Representación de las acciones necesarias para evaluar la pieza musical generada.}
\\
\hline \centerline{\textcolor[rgb]{1.00,1.00,1.00}{\textbf{\small
Actores}}} & {\small Usuario.}
\\
\hline
\begin{center}
\textcolor[rgb]{1.00,1.00,1.00}{\textbf{\small Casos de uso}}
\end{center}
\begin{center}

\end{center}
& {\small \emph{PuntuarPiezaMusical:} permite al usuario puntuar la pieza musical generada en base a su percepción subjetiva de pertenencia al género solicitado.}

{\small \emph{GenerarPiezaMusical:} permite al usuario solicitar la genearción de una pieza musical, dado un género concreto y descargarla.}

\\
\hline
\begin{center}
\end{center}
\begin{center}
\textcolor[rgb]{1.00,1.00,1.00}{\textbf{\small Flujo de eventos
principal}}
\end{center}
& {\small Pasos de ejecución del camino básico del caso de uso:}

{\small
\begin{enumerate}
    \item  El usuario solicita la creación de una pieza musical de un género concreto.

    \item  Una vez obtenida, el usuario puede emitir una valoración sobre la pertenencia de la pieza al género solicitado.
\end{enumerate}
}
\\
\hline \centerline{\textcolor[rgb]{1.00,1.00,1.00}{\textbf{\small
Flujo de eventos}}}
\centerline{\textcolor[rgb]{1.00,1.00,1.00}{\textbf{\small
excepcional}}} & {\small El paso número 2 es opcional.}
\\
\hline
\end{longtable}
\end{enumerate}
