\subsection{Diseño de la interfaz}

Como se ha mencionado en el apartado \emph{Especificación de requisitos de la interfaz} \ref{especificacion-interfaz}, la aplicación se compone de una única pantalla, basada en scroll vertical, que se irá adaptando a la acción que en cada momento se requiera al usuario.

Sobre ella serán cargados los paneles que se encarguen de la interacción \emph{usuario - máquina}.

En todo momento se ha perseguido hacer una interfaz en torno a la ``simpleza'' de uso y manejo de la aplicación, evitando controles, configuraciones y combinaciones que pudieran generar confusión.

A continuación se presentan los diseños de los paneles de la aplicación. Sendos paneles pueden verse recogidos en las imágenes de diseño \ref{fig:diseno-paneles}.

\subsubsection{Panel de selección de género musical}

En la figura \ref{fig:pantalla-seleccion} se muestra cómo quedaría el panel de selección musical maquetado.

% \begin{figure}[H]
% \begin {center}
%     \includegraphics[width=0.35\textwidth]{images/panel_seleccion.png}
%         \caption{Pantalla de selección de género musical.}
% \end {center}
% \label{fig:pantalla-seleccion}
% \end{figure}

Este panel albergará los siguientes items nombrados anteriormente en la especificación de la interfaz:

\begin{itemize}
    \item Una lista de géneros musicales para seleccionar.
    \item Un botón para instar la creación de una pieza musical del género seleccionado.
    \item Un botón para dirigirse a la zona de ayuda de la ventana.
\end{itemize}

La interacción con el panel de selección de género musical dará paso al siguiente estado de la pantalla.

El botón ``ayuda'' estará presente durante todo el ciclo de vida del software, haciendo que el scroll se desplace verticalmente, provocando que el \emph{View port} se mueva verticalmente y se pueda consultar el contenido de la ayuda.
 
\subsubsection{Panel de de descarga y valoración}

La figura \ref{fig:pantalla-descarga} muestra el diseño del panel que aparecerá como consecución de la solicitiud de una pieza musical. Permitirá descargar la pieza musical generada y emitir una valoración de la misma.

% \begin{figure}[H]
%     \begin {center}
%         \includegraphics[width=0.35\textwidth]{images/panel-descarga-valoracion.png}
%             \caption{Pantalla de descarga y valoración.}
%     \end {center}
%     \label{fig:pantalla-descarga}
%     \end{figure}

Al igual que en el panel anterior, el botón de ayuda estará visible e interactuable.

\begin{figure}[H]
  \centering
  \resizebox{0.7\textwidth}{!}{
    \begin{minipage}{\textwidth}
      \centering
      \begin{subfigure}[t]{0.4\textwidth}
          \centering
          \includegraphics[width=\textwidth]{images/panel_seleccion.png}
          \caption{Pantalla de selección de género musical.}
          \label{fig:pantalla-seleccion}
      \end{subfigure}
      \hfill
      \begin{subfigure}[t]{0.4\textwidth}
          \centering
          \includegraphics[width=\textwidth]{images/panel-descarga-valoracion.png}
          \caption{Pantalla de descarga y valoración.}
          \label{fig:pantalla-descarga}
      \end{subfigure}
    \end{minipage}
  }
  \caption{Diseño de paneles de pantalla.}
  \label{fig:diseno-paneles}
\end{figure}

