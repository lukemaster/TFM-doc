% INTRODUCCIÓN

\cleardoublepage

\chapter{Introducción}

Incluida en la clasificación de las bellas artes, en la lista propuesta por Ricciotto Canudo en su obra \emph{Manifiesto de las siete artes}, en la clasificación china de las tardes; y presente en cualquier obra audio visual contemporánea: cine, televisión, ``performances'', video juegos, ``video clips'', etc., la música es el elemento que marca una escena, condiciona una situación, acompaña una secuencia, potencia una acción, seduce a la audiencia, advierte al espectador o simplemente acompaña de fondo (y no por eso sin percibir menos atención) en la más mundana tarea como puede ser conducir un coche.

La música, en todos sus géneros y formas, relata las costumbres de una región, de una nación, de un país, la moda de un determinado momento, se hace eco de hechos históricos y es capaz de transferir conocimiento y saber popular entre generaciones.

Es una manera de expresarse. Una forma de expresión que modifica la manera de sentir, para seguir retroalimentando su presencia (quién no ha encontrado un patrón melódico en un llanto o en un ruido en plena calle).

La música, amplia, diversa e inmensa en número de piezas, consta de géneros musicales, que vienen a intentar clasificar y canonizar una composición musical, total o parcial, dentro de un grupo o conjunto, donde todos sus integrantes tengan rasgos comunes como: ritmo (compases y estructura), melodía, armonía, instrumentación, etc. En esencia, vienen a ser ``esas cosas que se perciben en conjunto y que hacen tipificar la pieza que se escucha''.

\section{Planteamiento del problema}

Tómese una serie de piezas musicales (cientos de ellas) y obténgase una representación computable de cada una. ¿Sería capaz una computadora de extraer patrones referenciales que permitieran clasificarlas? Esas clasificaciones, ¿se corresponderían con géneros musicales? ¿Cómo sabría la computadora de qué género musical se podría tratar en cada caso? Y si se añade una nueva pieza que concuerde levemente con alguna o algunas ya procesadas anteriormente, ¿será capaz de clasificarla? \textbf{¿Cómo aprende la máquina a distinguir géneros musicales?}

Esa es la gran pregunta que ha de hacerse para poder abordar la tarea de hacer un software que pueda clasificar piezas musicales.

Y una vez clasificadas, \textbf{¿podría la máquina generar nuevas piezas musicales que se correspondan con un género conocido?}

Ese en concreto es el problema que se aborda en este trabajo de fin de máster. Para ello, se van a explorar distintas formas de abordar el problema, contraponiendo los resultados de unas y otras, observando y evaluando qué alternativa cumple mejor con los requisitos planteados y cuál, además, aporta mejor gusto a la composición generada.


\section{Justificación}
La clasificación automática de géneros musicales aporta diversos valores. No se trata sólo de un desafío científico, también puede tener aplicaciones prácticas como:

\begin{itemize}
    \item \textbf{Sistemas de recomendación musical}: ser capaz de de clasificar de manera automática canciones por género. Es la base para construir sistemas de recomendación por género.
    \item \textbf{Bibliotecas musicales}: agrupar un conjunto de piezas musicales por géneros aportaría la capacidad de poder obtener conjuntos basados en características afines.
    \item \textbf{Análisis musicológico}: como herramienta para investigadores que necesiten estudiar la evolución de los géneros musicales. Analizar la influencia de distintos factores en la aparición de un género musical, evolución y rasgos comunes, etc.
    \item \textbf{Educación musical}: ayudar a estudiantes a distinguir entre géneros musicales y aprender algunos nuevos, combinarlos entre ellos, etc.
\end{itemize}

En este ámbito, el ocio y entretenimiento no se han quedado fuera. La Inteligencia Artificial ha transformando la manera en que se crean, distribuyen y consumen contenidos:

\begin{itemize}
    \item Se proyecta que para 2025, la IA gestionará el 95\% de las experiencias de los clientes en el sector del entretenimiento, mejorando la personalización y la eficiencia en la interacción \cite{marketers}.
    
    \item El 15\% de las interacciones globales en servicios de atención al cliente estarán impulsadas por IA para 2025 \cite{marketers}.
    
    \item Plataformas como Netflix (con más de 232 millones de usuarios activos mensuales) utilizan IA para predecir el crecimiento de suscriptores y optimizar la calidad de transmisión \cite{marketers}.
    
    \item En el campo de la música, OpenAI ha desarrollado \textit{Jukebox}, una red neuronal capaz de generar música en distintos géneros, incluyendo voz y letra \cite{jukebox}.
    
    \item Según un informe de CISAC, la IA podría reducir los ingresos globales del sector musical y audiovisual en un 20\% y 25\% respectivamente para 2028 \cite{elpais}.
    
    \item En cine, tecnologías de IA han sido utilizadas para rejuvenecer digitalmente a actores, como en \textit{The Irishman} de Martin Scorsese \cite{linkedin}.
    
    \item En videojuegos, la IA permite adaptar dinámicamente la dificultad del juego, personalizando la experiencia del usuario \cite{iaespana}.
    
    \item En el deporte, se emplean algoritmos de IA para analizar datos de partidos anteriores, predecir resultados y optimizar estrategias \cite{learningheroes}.
\end{itemize}

Así, la Inteligencia Artificial también aporta nuevas formas de creatividad y personalización, replanteando el papel de los humanos en la creación cultural.

\section{Objetivos}
\label{objetivo-principal}
Desarrollar un programa basado en Inteligencia Artificial que sea capaz de generar música personalizada, utilizando métodos generativos adversariales, que pueda ser clasificada dentro de un género concreto.

Para desarrollar el objetivo general, han de alcanzarse los objetivos específicos:

\begin{enumerate}
    \item Recopilar ejemplos de \textbf{datos} correspondientes con piezas musicales representativas de un género musical y que estén \textbf{bien etiquetadas}.
    
    \item Transformar esas piezas musicales en \textbf{estructuras de datos} analizables y computables.

    \item Diseñar un modelo basado en Inteligencia Artificial que sea capaz de \textbf{extraer información} crítica para \textbf{distinguir} cuáles son los \textbf{patrones} que asignan una pieza musical a un \textbf{género} concreto.

    \item Entrenar un modelo de Inteligencia Artificial, para que sea capaz de \textbf{generar} una pieza que sea reconocible como perteneciente a un género musical dado.

    \item \textbf{Evaluar} el modelo generador mediante métricas que se apliquen a los \textbf{datos de salida} generados.
\end{enumerate}

\section{Alcance}
Si se atiende al objetivo general, el alcance viene dado por la cantidad de géneros a seleccionar para poder producir piezas musicales distintas. Una pieza musical vendrá dada por un género y un ruido aleatorio introducido de manera automática, aspecto que limitaría el número máximo de piezas musicales distintas a producir por el sistema; es por eso que este ruido habrá de ser de gran dimensión.

\section{Estructura del documento}
Este documento se organiza de la siguiente manera:
\begin{itemize}
    \item \emph{Capítulo \textbf{1}}\textbf{: Introducción.} Presentación, exposición del problema y del trabajo que lo intenta resolver, justificación, objetivos, alcance y estructura (esta misma sección).
    \item \emph{Capítulo \textbf{2}}\textbf{: Estado del arte.} Revisión exhaustiva del estado del arte en el momento de acometer este trabajo fin de máster.
    \item \emph{Capítulo \textbf{3}}\textbf{: Marco teórico.} Exposición de los fundamentos teóricos en los que se basa este trabajo.
    \item \emph{Capítulo \textbf{4}}\textbf{: Métodos y Materiales.} Metodología utilizada y herramientas para llevar a cabo todo el proceso proyectual.
    \item \emph{Capítulo \textbf{5}}\textbf{: Resultados y Análisis.} Contraste de resultados y análisis de los mismos.
    \item \emph{Capítulo \textbf{6}}\textbf{: Conclusiones.} Presentación de conclusiones obtenidas del producto global del trabajo.
    \item \emph{Capítulo \textbf{7}}\textbf{: Limitaciones y perspectivas a futuro.} Propuestas de mejoras y líneas de expansión para futuros trabajos.
\end{itemize}