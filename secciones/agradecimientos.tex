% AGRADECIMIENTOS

\cleardoublepage

\normalfont{\huge{\bfseries{Agradecimientos}}}
\vspace{15ex}

% Escribe tus agradecimientos a continuación.
% Se recomienda separar cada párrafo con un \newline.

En primer lugar, me gustaría mostrar agradecimiento a todas las personas que han compartido mi día a día durante un duro e intenso (algo más de un) año.
\newline
Este trabajo, fruto de mi esfuerzo, dedicación y mejor saber hacer, se lo dedico expresamente a:
\newline
Mi familia, papá, mamá y Milo, porque han sabido sacrificar su tiempo conmigo para que yo pudiera estudiar y hacer deberes.
\newline
Sonia, esa sombra opaca, fuerte, ruda y erguida, como una pared sólida en la que apoyarse siempre.
\newline
Equipo de Astroprint, porque de ellos he aprendido muchas de las cosas que me han sido muy útiles en este curso.
\newline
Ángel y Alejandro, porque han sabido poner la palabra exacta en el momento preciso y siempre me han hecho mirar hacia adelante cuando sólo podía mirar hacia abajo.
\newline
A todos mis compañeros del máster, porque este trabajo es también parte de ellos.
Quiero hacer una mención especial a: Eva, Adolfo, José, Pelayo, Sandra, Cristi, Karina, Diana, Lidia, Roberto y Guillermo; esos nombres que siempre estaban en chat.
Y una mención muy muy especial a Luis Soto Medina, que demuestra que la importancia del camino es tanta o más, que llegar al destino; y si de algo me quedo de este camino, es su amistad.
\newline
Yaneth Coromoto, porque ha sabido siempre dirigir el rumbo y dar los toques y matices necesarios para que yo pudiera seguir caminando. No sé si se han encontrado alguna vez a una de esas personas de las que uno siente que puede aprender muchísimo y de las que les gustaría no desvincularse... pues he aquí una de ellas. ¡¡Muchas gracias!!
\newline
\newline
Quiero hacer una mencioncita pequeña a los dos locos que han estado conmigo incondicionalmente durante todo este tiempo. En los peores momentos siempre han tenido un ``miau'' y/o un cabezazo para mí. Gracias, H.M. Murdock y Miso (alias ``el Quicu'').
\newline
\newline
Por último, la más especial de las dedicatorias, a mi esposa, Cristina. Sin ella, no hubiera sido posible. Le agradeceré siempre que pilotara sola, durante las últimas semanas, la nave que lleva mucho tiempo, pilotando casi sola. Gracias por hacer que todo siguiera funcionando igual durante mi ausencia a los mandos, desde hace algo más de un año, para poder dedicar tiempo a esto. Si existiera el reconocimiento de ``consorte'' en el título del máster, sin duda, sería para mí y ella obtendría la titulación.
