\cleardoublepage

\chapter*{Resumen}
\label{resumen}
\addcontentsline{toc}{chapter}{Resumen}

La Inteligencia Artificial ha evolucionado enormemente desde su aparición a finales de la década de 1950 y su posterior auge en las décadas de los 70 y 80 del siglo pasado, con los sistemas basados en reglas y el razonamiento aproximado. El nuevo auge de la Inteligencia Artificial ha traído consigo los modelos generativos. La generación de música es una aplicación emergente dentro esos nuevos modelos, sustentada en Autoencoders Variacionales (VAE) y en las sofisticadas Redes Generativas Adversariales (GANs). Esta línea de trabajo se ha potenciado gracias a la creciente disponibilidad de datos digitales musicales y a la madurez de las arquitecturas neuronales aplicadas al procesamiento de espectrogramas.

Este trabajo de fin de máster, aborda la generación de música personalizada condicionada por género musical, combinando distintos enfoques generativos en un sistema que permite la evaluación de piezas musicales ``artificialmente'' generadas. El proceso se articula a través de un \textit{pipeline} estructurado que abarca desde la recopilación y transformación de archivos de audio en representaciones espectrales, hasta el entrenamiento de modelos generativos adversariales capaces de reconstruir y generar música desde ruido latente y género explicitado.

Inicialmente, se desarrollan modelos basados en VAE para explorar la compresión y reconstrucción de espectrogramas, con el objetivo de capturar estructuras musicales coherentes. Posteriormente, se integra un sistema GAN con arquitectura híbrida basada en Transformers, orientado a mejorar la capacidad creativa y la calidad perceptiva de las piezas generadas. Ambos enfoques se complementan en su búsqueda de una síntesis musical más flexible y significativa.

La combinación de técnicas ágiles de desarrollo y lenguajes y frameworks de modelo software, dan como resultado un sistema que aúna técnicas generativas, representación espectral y condicionamiento por género.